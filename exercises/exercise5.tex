\section*{Recomendaciones}
\addcontentsline{toc}{section}{Recomendaciones}
Luego del análisis hecho en este trabajo, se entregan las siguientes recomendaciones para poder reducir los índices de accidentalidad en Colombia:
\begin{itemize}
    \item Dada la relación entre el número de pasajeros y los accidentes reportados, se puede recomendar viajar acompañados para viajes largos. En este sentido lo que se recomienda es generar campañas para evitar viajar solo durante trayectos largos, implementar leyes de transporte para que se pueda implementar esto en conductores profesionales.
    \item Dar más capacitaciones sobre la normatividad a los conductores, sobre todo teniendo en cuenta su nivel educativo, junto con programas de senbilización y responsabilización al momento de conducir.
    \item Promover una ley para exigir un grado escolar mínimo al momento de aplicar a una licencia de conducir.
    \item De acuerdo a lo que se dijo durante el análisis en \ref{sec:propuestaSolucion}, donde se habla de cómo las emociones podrían ser un factor en los índices de accidentalidad en Colombia, se deben promover cursos o capacitaciones espacializados en el manejo de las emociones durante la conducción.
    \item Implementar mayores estudios y análisis por parte de las entidades gubernamentales, para detectar causas, y poder solventar las problemáticas de accidentalidad en Colombia.
\end{itemize}