\section*{Propuesta de solución a la problemática}
\label{sec:propuestaSolucion}
\addcontentsline{toc}{section}{Propuesta de solución a la problemática}
Una vez analizadas algunas de las variables de los registros de accidentalidad en Colombia, se presentan algunas interpretaciones con el fin de encontrar algunas causas posibles de accidentalidad.\par
De las variables cualitativas podemos concluir que, de acuerdo a la tabla \ref{tab:tablaFrecuentasVariableCualitativa} la mayoría de accidentes registrados son causados por conductores con secundario completo. De todas formas es interesante notar que el segundo grupo de conductores con más tasa de accidentalidad son personas sin educación formal. Esto último puede indicar que el nivel educativo tiene una inherencia en los índices de accidentes.\par
De la tabla de contingencia \ref{tab:tablaContingencia} se obtiene que la mayoría de los accidentes son causados por hombres. También podemos ver que la mayoría de los hombres que causan accidentes están mejor preparados que las mujeres que causan accidentes. Si bien es esperable que una persona que no esté bien educada pueda ser más propensa a tener accidentes por desconocimiento de las normatividad vial, o por falta de conciencia social, es interesante ver que en el caso de los hombres, esto parece no afectar. Una posible explicación de este hecho es que, en primer lugar, en general existen más hombres conductores que mujeres conductoras, por lo que esto puede afectar las mediciones. También se puede deber a que los hombres manejan de manera diferente situaciones de estrés, prisas, y por que en general pueden ser más “competitivos” generando situaciones de peligro.\par
Analizando la variable \emph{Número de lesionados} de la tabla \ref{tab:variablesDiscretasMedidasCentral} se puede ver que el número de lesionados varía entre ninguno, y seis. En promedio se presentan casi tres lesionados por accidente en Colombia pero, según la moda, en la mayoría no se presenta ninguno, es decir que cuando se presenta un accidente en Colombia, el número de lesionados es alto o muy bajo. En este último caso, se pueden dar otras consecuencias, como la destrucción del vehículo de transporte o terminar directamente con la muerte de los ocupantes.\par
Del análisis de la variable \emph{Número de ocupantes}, de acuerdo a lo mostrado en la gráfica \ref{fig:diagramasBarrasVariablesDiscretas}, podemos concluir que en la mayoría de los casos los accidentes ocurren cuando sólo está el conductor. El hecho de que entre más ocupantes haya menos registros de accidentes se puede deber en primer lugar a que gracias a la compañía se disminuyen algunas situaciones como el microsueño. Otra posible causa de esta tendencia es que entre más pasajeros se lleven en el vehículo, el conductor toma mayor responsabilidad sobre la vida y bienestar de estos, siendo más precavido y cauto al conducir.\par
Si se analiza la variable continua \emph{kilómetros recorridos por el vehículo}, se puede deducir que la distancia máxima recorrida antes de un accidente fue de 112 kilómetros. Hay que tener presente que las distancias registradas en los registros pueden ser de accidentes dentro de centros urbanos, o en carreteras, donde las distancias recorridas normalmente son mayores. Teniendo en cuenta los valores de distancia máximo y mínimo encontrados se puede decir que los registros se corresponden con viajes intermunicipales, pues son bastante más grandes que las distancias que se recorren normalmente dentro de una ciudad.\par
Finalmente, al estudiar la relación entre las variables \emph{Edad del conductor} y \emph{Velocidad del vehículo}, hay que tener en cuenta que cada par de datos $(t, v(t))$ es parte de otro conjunto de datos relacionados a un tipo de accidente particular, por lo que las relaciones encontradas entre la edad del conductor y la velocidad pueden indicar algo como que se necesita que el conductor sea más joven para que un accidente pueda ocurrir a menor velocidad. Esto puede tener sentido, ya que los conductores jovenes son inexpertos y generar accidentes más facilmente por otros factores diferentes al exceso de velocidad, como falta de atención, exceso de confianza causado por la misma edad e inmadurez mental, o temor a realizar algunas maniobras. Los conductores mayores por otra parte pueden evitar estos problemas, y los accidentes relacionados a conductores mayores pueden deberse más a exceso de velocidad por una combinación de exceso de confianza por acostumbramiento, con perdida de reflejos, o de visión y audición por la edad.