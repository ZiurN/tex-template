\section*{Propuesta de solución a la problemática}
\addcontentsline{toc}{section}{Propuesta de solución a la problemática}
\begin{quotation}
    \emph{\textcolor{gray}{Finalmente, el grupo deberá responder a la pregunta: ¿Qué alternativa de solución plantea para la problemática estudiada?, dicha respuesta deberá estar justificada descriptivamente, es decir para ello (utilizará tablas, gráficos, medidas, diagramas, entre otros) a partir de los resultados estadísticos descriptivos realizados en la actividad anterior (100 primeros datos) y la información obtenida de la problemática planteada.}}\par
    \emph{\textcolor{gray}{NOTA: Describir es explicar, representar, definir con detalle, las cualidades características o las circunstancias de algo o de alguien.}}\par
    \emph{\textcolor{gray}{Cabe aclarar que la solución de la problemática no debe ser una cuestión subjetiva, y debe incluir la información disponible en los análisis estadísticos realizados.}}\par
    \emph{\textcolor{gray}{Por ejemplo:}}\par
    \emph{\textcolor{gray}{Del diagrama estadístico “XX" se desprende que.... De la tabla "XX" se desprende que.... A partir del valor de las medidas univariantes o de dispersión …Podemos afirmar que…. y… por esto consideramos necesario realizar……. Para ello es necesario disponer de …. (ver tabla XX) y se propone realizar…….}}
\end{quotation}
Esta actividad se debe completar con los análisis ya hechos, va por parte de todos pues es necesario redactar los análisis encontrados individualmente.