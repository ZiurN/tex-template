\section*{Regresión y correlación entre variables cuantitativas}
\addcontentsline{toc}{section}{Regresión y correlación entre variables cuantitativas}
Para el desarrollo de esta actividad, se toma como variable independiente la \textbf{\emph{Edad del conductor}}, como variable dependiente la \textbf{\emph{Velocidad del vehículo}}.\par
En la figura \ref{fig:scatterPlotExercise} se muestra la gráfica de dispersión de la \emph{velocidad del vehículo} en relación con la \emph{edad del conductor}.
\begin{figure}[!ht]
    \centering
    {\footnotesize
\begin{tikzpicture}
    \begin{axis}[
        width=.6\linewidth,
        title={Relación entre la edad del conductor y la velocidad del vehículo},
        grid style={line width=.1pt, draw=gray!10},
        xminorgrids=true,
        yminorgrids=true,
        minor tick num=5,
        enlargelimits=true,
        xlabel=Edad del conductor (Años),
        ylabel=Velocidad del vehículo ($\frac{km}{h}$)
    ]
    \addplot[only marks,  mark=o] table{assets/scatterPlotExercise.dat};
    \end{axis}
\end{tikzpicture}}
    \caption{\footnotesize{Diagrama de dispersión de la velocidad del vehículo en relación a la edad del conductor.}}
    \label{fig:scatterPlotExercise}
\end{figure}
De acuerdo al diagrama de dispersión, se deduce que la relación entre la \emph{velocidad del vehículo} en relación con la \emph{edad del conductor} es \textbf{lineal positiva}.
Al realizar el análisis realizado en Excel con los datos, el \textbf{coeficiente de determinación es igual a $R^{2}=0.6133$}. El \textbf{coeficiente de correlación es igual a $R=0.783$}\footnote[1]{Aquí, una vez más, el valor de $R$ se determina no sólo con el cálculo de la raíz cuadrada del valor del coeficiente de determinación $R^{2}$, sino de la relación observada en la gráfica \ref{fig:scatterPlotExercise}}. Para estos datos el modelo de regresión lineal encontrado para la relación entre la edad del conductor $t$ y la velocidad del vehículo $v$ es
\begin{equation}
    \label{eq:modelo1}
    v(t) = 0.8996\cdot t + 21.843
\end{equation}
Este modelo no tiene una gran confianza, ya que la correlación se puede considerar \emph{regular} ya que el coeficiente de determinación en su versión porcentual es de $61.33\%$. En la figura \ref{fig:scatterPlotExerciseb} se puede apreciar la recta de regresión obtenida a partir de los datos entregados, junto con los valores de dispersión correspondientes.\par
\begin{figure}[!ht]
    \centering
    {\footnotesize
\begin{tikzpicture}
    \begin{axis}[
        width=.6\linewidth,
        title={Relación entre la edad del conductor y la velocidad del vehículo},
        grid style={line width=.1pt, draw=gray!10},
        xminorgrids=true,
        yminorgrids=true,
        minor tick num=5,
        enlargelimits=true,
        xlabel=Edad del conductor (Años),
        ylabel=Velocidad del vehículo ($\frac{km}{h}$),
        legend pos=south east
    ]
    \addplot[chartColor0, dotted, domain=20:91, line width=2pt] {0.8996*x+21.843};
    \addlegendentry{$v(t) = 0.8996t + 21.843$}
    \addplot[only marks,  mark=o] table{assets/scatterPlotExercise.dat};
    \end{axis}
\end{tikzpicture}}
    \caption{\footnotesize{Diagrama de dispersión de la temperatura en función de la frecuencia cardiaca junto a la recta aproximada resultado de la regresión lineal aplicada.}}
    \label{fig:scatterPlotExerciseb}
\end{figure}
Para relacionar estos resultados con la problemática de accidentalidad en Colombia, hay que tener en cuenta que cada par de datos $(t, v(t))$ es parte de otro conjunto de datos relacionados a un tipo de accidente particular, por lo que las relaciones encontradas entre la edad del conductor y la velocidad pueden indicar algo como que se necesita que el conductor sea más joven para que un accidente pueda ocurrir a menor velocidad. Esto puede tener sentido, ya que los conductores jovenes son inexpertos y generar accidentes más facilmente por otros factores diferentes al exceso de velocidad, como falta de atención, exceso de confianza causado por la misma edad e inmadurez mental, o temor a realizar algunas maniobras. Los conductores mayores por otra parte pueden evitar estos problemas, y los accidentes relacionados a conductores mayores pueden deberse más a exceso de velocidad por una combinación de exceso de confianza por acostumbramiento, con perdida de reflejos, o de visión y audición por la edad.