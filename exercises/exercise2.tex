\section*{Caracterización de variables cuantitativas}
\addcontentsline{toc}{section}{Caracterización de variables cuantitativas}
\subsection*{Variables cuantitativas discretas}
\addcontentsline{toc}{subsection}{Variables cuantitativas discretas}
En la tabla \ref{tab:variablesDiscretasMedidasCentral} se entregan los datos y medidas de tendencia central solicitados en la actividad para las dos variables discretas \emph{Numero de lesionados} y \emph{Número de ocupantes}. Así mismo se entregan en la tabla \ref{tab:variablesDiscretasMedidasDispersion} las correspondientes medidas de dispersión.\par
\begin{table}[!htbp]
    \begin{footnotesize}
        \centering
        \begin{tabular}{lc|c}
            \toprule
            {} & \textbf{Numero de lesionados} & \textbf{Número de ocupantes} \\
            \bottomrule
            \toprule
            \multicolumn{3}{l}{\textbf{Medidas de tendencia central}} \\
            \bottomrule
            Promedio & 2.96 & 2.68 \\
            \midrule
            Moda & 0 & 1\\
            \midrule
            Mediana & 3 & 2 \\
            \bottomrule
            \toprule
            \multicolumn{3}{l}{\textbf{Medidas de posición}} \\
            \bottomrule
            Cuartil 1 & 1 & 1 \\
            \midrule
            Cuartil 2 & 3 & 2 \\
            \midrule
            Cuartil 3 & 5 & 4 \\
            \bottomrule
        \end{tabular}
        \caption{\footnotesize{Medidas de tendencia central y de posición para las variable cuantitativas discretas  \emph{Numero de lesionados} y \emph{Número de ocupantes}.}}
        \label{tab:variablesDiscretasMedidasCentral}
    \end{footnotesize}
\end{table}
\begin{table}[!htbp]
    \begin{footnotesize}
        \centering
        \begin{tabular}{lc|c}
            \toprule
            {} & \textbf{Numero de lesionados} & \textbf{Número de ocupantes} \\
            \bottomrule
            \toprule
            \multicolumn{3}{l}{\textbf{Medidas de dispersión}} \\
            \bottomrule
            \toprule
            Valor mínimo & 0 & 1 \\
            \midrule
            Valor máximo & 6 & 8 \\
            \midrule
            Rango & 6 & 7 \\
            \midrule
            Varianza & 4.52 & 3.11 \\
            \midrule
            Desviación Estándar & 2.13 & 1.76 \\
            \midrule
            Coeficiente de variación & 0.72 & 0.65 \\
            \bottomrule
        \end{tabular}
        \caption{\footnotesize{Medidas de dispersión para las variables cuantitativas discretas  \emph{Numero de lesionados} y \emph{Número de ocupantes}.}}
        \label{tab:variablesDiscretasMedidasDispersion}
    \end{footnotesize}
\end{table}
\newpage
Finalmente se muestran los diagramas de barras correspondientes en la figura \ref{fig:diagramasBarrasVariablesDiscretas}.\par
\begin{figure}[!ht]
    \centering
    \noindent\parbox[][][c]{.5\linewidth}{
        \scriptsize{
    \pgfplotsset{every x tick label/.append style={font=\tiny}}
    \begin{tikzpicture}
        \begin{axis}[
            xlabel={Número de Lesionados},
            ylabel={Frecuencia},
            minor tick num=0,
            xtick = {0,1,2,3,4,5,6},
            xmin=-1, xmax=7,
            ymin=9, ymax=18,
            bar width=15pt,
            ymajorgrids,
            enlargelimits=0.05,
            nodes near coords,
            nodes near coords align={vertical},
          ]
          \addplot [ybar, fill=chartColor5, draw=none] table {./assets/numeroLesionados.dat};
        \end{axis}
    \end{tikzpicture}
}
    }
    \parbox[][][c]{.4\linewidth}{
        \scriptsize{
    \pgfplotsset{every x tick label/.append style={font=\tiny}}
    \begin{tikzpicture}
        \begin{axis}[
            xlabel={Número de Ocupantes},
            minor tick num=5,
            xtick = {1, 2, 3, 4, 5, 6, 8},
            xmin=0, xmax=9,
            ymin=1.5, ymax=35,
            bar width=15pt,
            ymajorgrids,
            enlargelimits=0.05,
            yminorgrids=true,
          ]
          \addplot [ybar, fill=chartColor5, draw=none] table {./assets/numeroOcupantes.dat};
        \end{axis}
    \end{tikzpicture}
}
    }\par
    \caption{\footnotesize{Diagramas de barras para las variables discretas \emph{Número de lesionados} y \emph{Número de ocupantes}.}}
    \label{fig:diagramasBarrasVariablesDiscretas}
\end{figure}
\subsection*{Variable cuantitativa continua}
\addcontentsline{toc}{subsection}{Variable cuantitativa continua}
Para el análisis univariante de variable cuantitativa continua se ha tomado la variable \emph{kilómetros recorridos por el vehículo}. En la tabla \ref{tab:variableContinuaMedidasCentrales} se muestran las medidas de tendencia central de dicha variable, así como las medidas de dispersión en la tabla \ref{tab:variableContinuaMedidasDispersion}.\par
\vspace{2em}
\begin{footnotesize}
    \begin{minipage}[!ht][9cm][b]{0.45\textwidth}
        \centering
        \begin{tabular}{lc}
            \toprule
            \multicolumn{2}{l}{\textbf{Medidas de tendencia central}} \\
            \bottomrule
            Promedio & 68.78\\
            \midrule
            Moda & 35.80\\
            \midrule
            Mediana & 63.70\\
            \bottomrule
            \toprule
            \multicolumn{2}{l}{\textbf{Medidas de posición}} \\
            \bottomrule
            Cuartil 1 & 49.85 \\
            \midrule
            Cuartil 2 & 63.70 \\
            \midrule
            Cuartil 3 & 90.75 \\
            \bottomrule
        \end{tabular}
        \captionof{table}{\footnotesize{Medidas de tendencia central para la variable cuantitativa continua \emph{kilómetros recorridos por el vehículo}.}}
        \label{tab:variableContinuaMedidasCentrales}
    \end{minipage}
    \begin{minipage}[!ht][9cm][b]{0.5\textwidth}
        \centering
        \begin{tabular}{lc}
            \toprule
            \multicolumn{2}{l}{\textbf{Medidas de dispersión}} \\
            \bottomrule
            \toprule
            Valor mínimo & 26.40 \\
            \midrule
            Valor máximo & 112 \\
            \midrule
            Rango & 84.00 \\
            \midrule
            Varianza & 606.98 \\
            \midrule
            Desviación Estándar & 24.64 \\
            \midrule
            Coeficiente de variación & 0.36 \\
            \bottomrule
            {} & {} \\
        \end{tabular}
        \captionof{table}{\footnotesize{Medidas de dispersión para la variable cuantitativa continua \emph{kilómetros recorridos por el vehículo}.}}
        \label{tab:variableContinuaMedidasDispersion}
    \end{minipage}
\end{footnotesize}
\par