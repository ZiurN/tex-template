\section*{Caracterización de variables cuantitativas, discretas y continuas}
\addcontentsline{toc}{section}{Caracterización de variables cuantitativas, discretas y continuas}
\begin{quotation}
    \emph{\textcolor{gray}{Para una de las variables discreta elegida, se deberán calcular las medidas univariantes de tendencia central: media, mediana y moda. Todos los cuartiles. Así mismo deberán calcular las medidas univariantes de dispersión: rango, varianza, desviación típica y coeficiente de variación.}}\par
    \emph{\textcolor{gray}{Para una de las variables continúa elegida, se deberán calcular las medidas univariantes de tendencia central: media, mediana y moda. Todos los cuartiles. Así mismo deberán calcular las medidas univariantes de dispersión: rango, varianza, desviación típica y coeficiente de variación.}}
\end{quotation}
Esta actividad esta a cargo de Cristhian para la variable discreta \emph{Número de lesionados}, Jeferson a cargo de la variable discreta \emph{Número de ocupantes}, y Juana a cargo de la variable continua \emph{Kilometros recorridos por vehículo}.


