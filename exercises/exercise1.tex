\section*{Caracterización de variables cualitativas}
\addcontentsline{toc}{section}{Caracterización de variables cualitativas}
Para la primera parte del análisis se selecciona la variable cualitativa \emph{Nivel educativo}. En la tabla \ref{tab:tablaFrecuentasVariableCualitativa} se entregan las frecuencias asociadas a la variable seleccionada.\par
\begin{table}[!htbp]
    \begin{footnotesize}
    \centering
    \begin{tabular}{lccc}
        \toprule
        \multicolumn{1}{c}{\textbf{Categorías}} & \multicolumn{1}{c}{\textbf{Frecuencia absoluta}} & \multicolumn{1}{c}{\textbf{Frecuencia relativa}} & \multicolumn{1}{c}{\textbf{Frecuencia porcentual}}\\
        \midrule
        Sin educación & 18 & 0.18 & 18\% \\
        \midrule
        Primaria completa & 12 & 0.12 & 12\% \\
        \midrule
        Secundaria completa & 24 & 0.24 & 24\% \\
        \midrule
        Técnico/Tecnólogo & 15 & 0.15 & 15\% \\
        \midrule
        Universitaria & 18 & 0.18 & 18\% \\
        \midrule
        Posgrado & 13 & 0.13 & 13\% \\
        \toprule
        \rowcolor{myGray}
        \textbf{Totales} & \textbf{100} & \textbf{1.00} & \textbf{100\%} \\
        \bottomrule
    \end{tabular}
    \caption{\footnotesize{Tabla de frecuencias para la variable \emph{Nivel educativo}.}}
    \label{tab:tablaFrecuentasVariableCualitativa}
    \end{footnotesize}
\end{table}
En la figura \ref{fig:graficasVariableCualitativa} se presentan el gráfico de barras y el diagrama circular para la variable \emph{El tipo de vehículo}.\par
\vspace{1em}
\begin{figure}[!ht]
    \centering
    \noindent\parbox[][][c]{.55\linewidth}{
        \scriptsize{
\begin{tikzpicture}
    \begin{axis}[
                bar width=10pt,
                enlargelimits=0.05,
                ymajorgrids,
                ymax=25,
                ymin=10,
                ylabel={Número de accidentes de tránsito},
                symbolic x coords={Sin educación, Primaria completa, Secundaria completa, Técnico/Tecnólogo, Universitaria, Posgrado},
                xtick={Sin educación, Primaria completa, Secundaria completa, Técnico/Tecnólogo, Universitaria, Posgrado},
                xticklabel style={rotate=45,anchor=north east},
                xtick align=inside,
                nodes near coords,
                nodes near coords align={vertical},
            ]
    \addplot[ybar, fill=chartColor0, draw=myGray] coordinates {(Sin educación, 18)};
    \addplot[ybar, fill=chartColor1, draw=none] coordinates {(Primaria completa, 12)};
    \addplot[ybar, fill=chartColor2, draw=none] coordinates {(Secundaria completa, 24)};
    \addplot[ybar, fill=chartColor3, draw=none] coordinates {(Técnico/Tecnólogo, 15)};
    \addplot[ybar, fill=chartColor4, draw=none] coordinates {(Universitaria, 18)};
    \addplot[ybar, fill=chartColor5, draw=none] coordinates {(Posgrado, 13)};
    \end{axis}
\end{tikzpicture}
}
    }
    \parbox[][][c]{.4\linewidth}{
        \scriptsize{
    \begin{tikzpicture}
        \pie[
            rotate=180,
            text=inside,
            color={chartColor0,chartColor1,chartColor2,chartColor3,chartColor4,chartColor5},
            draw=myGray,
            explode = 0.1
        ]
         {18/Sin educación, 24/Secundaria completa, 12/Primaria completa, 18/Universitario, 15/Técnico-Tecnólogo, 13/Posgrado}
    \end{tikzpicture}
}

    }\par
    \caption{\footnotesize{Diagrama de barras y circular para las frecuencias de la variable \emph{Nivel educativo}.}}
    \label{fig:graficasVariableCualitativa}
\end{figure}
\begin{mdframed}[hidealllines=true,backgroundcolor=myGray,skipabove=5pt]
    La moda para el \emph{Nivel educativo} es \emph{Secundaria completa} con un valor de $24$.
\end{mdframed}
\begin{table}[!htbp]
    \centering
    \begin{footnotesize}
    \begin{tabular}{ l x{0.1\linewidth} x{0.1\linewidth} y{0.07\linewidth} }
        \toprule
        {} & \multicolumn{2}{c}{\textbf{Sexo del conductor}} & {} \\ \cline{2-3}
        \multirow{-2}{*}{\textbf{\begin{tabular}{c}Nivel\\educativo\end{tabular}}} & \emph{\begin{tabular}{c}Femenino\end{tabular}} & \emph{\begin{tabular}{c}Masculino\end{tabular}} & \multirow{-2}{*}{\cellcolor{myGray}\textbf{Totales}}\\
        \bottomrule
        Sin educación & 6 & 12 & \textbf{18} \\
        \midrule
        Primaria completa & 8 & 4 & \textbf{12} \\
        \midrule
        Secundaria completa & 9 & 15 & \textbf{24} \\
        \midrule
        Técnico\/Tecnólogo & 4 & 11 & \textbf{15} \\
        \midrule
        Universitaria & 6 & 12 & \textbf{18} \\
        \midrule
        Posgrado & 4 & 9 & \textbf{13} \\
        \toprule
        \rowcolor{myGray}
        \textbf{Totales} & \textbf{37} & \textbf{63} & \textbf{100}\\
        \bottomrule
        \end{tabular}
    \caption{\footnotesize{Tabla de contingencia de las variables \emph{Nivel educativo} y \emph{Sexo del conductor}.}}
    \label{tab:tablaContingencia}
    \end{footnotesize}
\end{table}