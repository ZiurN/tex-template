\section*{Objetivos}
\addcontentsline{toc}{section}{Objetivos}
\subsection*{Objetivo General}
\addcontentsline{toc}{subsection}{Objetivo General}
Proponer alternativas de solución a la problemática, a través de un informe descriptivo de las técnicas estadísticas implementadas en la base de datos, obteniendo así un aprendizaje básico, la cual nos permite interpretar, calcular y analizar los diferentes estudios estadísticos que se presentan en nuestra vida diaria .
\subsection*{Objetivos Específicos}
\addcontentsline{toc}{subsection}{Objetivos Específicos}
	\begin{itemize}
		\item Calcular medidas de tendencia central (como la media, la mediana y la moda) para describir la distribución de variables relacionadas con la accidentalidad vial.
		\item Determinar medidas de dispersión (como la varianza y la desviación estándar) para evaluar la variabilidad de los datos.
		\item Evaluar la relación entre variables (por ejemplo, velocidad, tipo de vehículo, condiciones climáticas) y la frecuencia de accidentes.
		\item Plantear una solución para la problemática y adquirir conocimientos sobre las operaciones estadísticas.
	\end{itemize}
\newpage