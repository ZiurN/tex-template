\section*{Justificaci\'on}
\addcontentsline{toc}{section}{Justificaci\'on}
Con el desarrollo de la presente actividad teniendo como referencia la base de datos de  Anexo 1 - Indicadores de accidentalidad vial en Colombia para el primer semestre de 2023 (16-4), trabajada en el periodo, se realiza un ajuste dejando los 100 primeros datos,  luego a partir de esto se toma como mínimo seis variables (al menos dos cualitativas, dos cuantitativas discretas y dos cuantitativas continuas) lo que se busca es poder facilitar el aprendizaje, en el  manejo de las variables cualitativas y cuantitativas, así como  su interpretación.\par
La correlación y regresión son conceptos estadísticos fundamentales, que muestran la idea de dependencia funcional, relacionándose de formas diversas como la variación, distribución, centralización o dispersión. Es a través de este saber estadístico que el estudiante interactúa con áreas de investigación tales como son la parte esencial de su campo profesional fortaleciendo así su perfil de formación dado que lo ayudará a una toma inteligente de decisiones.\par
Con este estudio se puede ratificar los procedimientos estadísticos que se utiliza para el propósito de descripción con el fin de organizar los datos, colocando en función los estudios de esos diferentes tipos de presentación estadística. Se refleja la efectividad con la que se puede realizar el proceso de información dependiendo de la presentación de dichos datos, puntualmente siendo la gráfica unas de las formas más rápidas y precisas de analizar resultados.